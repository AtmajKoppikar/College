% me=0 student solutions (ps file), me=1 - my solutions (sol file),
% me=2 - assignment (hw file)
\def\me{0} \def\num{11} %homework number

\def\due{5 pm on Tuesday, December 05} %due date

\def\course{CSCI-GA.1170-001, 003 Fundamental Algorithms} %course name, changed only once

% **** INSERT YOUR NAME HERE ****
\def\name{Name}

% **** INSERT YOUR NETID HERE ****
\def\netid{NetID}

% **** INSERT NETIDs OF YOUR COLLABORATORS HERE ****
\def\collabs{NetID1, NetID2}


\iffalse

INSTRUCTIONS: replace # by the homework number.  (if this is not
ps#.tex, use the right file name)

Clip out the ********* INSERT HERE ********* bits below and insert
appropriate LaTeX code.  There is a section below for student macros.
It is not recommended to change any other parts of the code.


\fi
%

\documentclass[11pt]{article}

% ==== Packages ====
\usepackage{amsfonts,amsmath,amssymb,amsthm}
\usepackage{mathtools}
\usepackage{latexsym}
\usepackage{etoolbox}
\usepackage{totcount}
\usepackage{fullpage}
\usepackage{graphicx}
\usepackage{tikz}
\usepackage{tikz-qtree}
\usepackage[bottom]{footmisc}
\usepackage{enumitem}
\usepackage{hyperref}

\usepackage{array}
\newcolumntype{C}[1]{>{\centering\let\newline\\\arraybackslash\hspace{0pt}}m{#1}}
\newcolumntype{R}[1]{>{\raggedright\let\newline\\\arraybackslash\hspace{0pt}}m{#1}}

\usetikzlibrary{arrows}

\setlength{\footskip}{1in} \setlength{\textheight}{8.5in}

\newcommand{\handout}[5]{
	\renewcommand{\thepage}{#1, Page \arabic{page}}
	\noindent
	\begin{center}
		\framebox{ \vbox{ \hbox to 5.78in { {\bf \course} \hfill #2 }
				\vspace{4mm} \hbox to 5.78in { {\Large \hfill #5 \hfill} }
				\vspace{2mm} \hbox to 5.78in { {\it #3 \hfill #4} }
				\ifnum\me=0
				\vspace{2mm} \hbox to 5.78in { {\it Collaborators: \collabs
						\hfill} }
				\fi
		} }
	\end{center}
	\vspace*{4mm}
}

\newcounter{pppp}
\newcounter{pppc}
\newcommand{\prob}{\arabic{pppp}} %problem number
\newcommand{\increase}{\stepcounter{pppp}\stepcounter{pppc}} %problem number

% Arguments: Title, Number of Points
\newcommand{\newproblem}[2]{
	\ifnum\me=0
	\ifnum\prob>0 \newpage \fi
	\increase
	\restartlist{subtasks}
	\setcounter{page}{1}
	\handout{\name{} (\netid), Homework \num, Problem \arabic{pppp}}
	{\today}{Name: \name{} (\netid)}{Due: \due}
	{Solutions to Problem \prob\ of Homework \num\ (#2)}
	\else
	\increase
	\restartlist{subtasks}
	\section*{Problem \num-\prob~(#1) \hfill {#2}}
	\fi
}

\newlist{subtasks}{enumerate}{1}
\setlist[subtasks]{label={(\alph*)},resume}

\newcounter{numpppp}
\loop
\stepcounter{numpppp}
% workaround bug in totcount (means \newtotcounter{points\arabic{pointsct}})
\begingroup%
\edef\tempcounter@@name{points\arabic{numpppp}}%
\expandafter\newtotcounter\expandafter{\tempcounter@@name}%
\edef\tempcounter@@name{ecpoints\arabic{numpppp}}%
\expandafter\newtotcounter\expandafter{\tempcounter@@name}%
\endgroup
\ifnum \value{numpppp}<500 % max number of tasks supported
\repeat

% formating of output
\newcommand{\disppoints}[1]{%
	\texorpdfstring{(#1~\ifnumequal{#1}{1}{point}{points})}{}
}

% adds and displays
\newcommand{\points}[1]{%
	\texorpdfstring{\addtocounter{points\arabic{pppc}}{#1}\disppoints{#1}}{}%
}
\newcommand{\ecpoints}[1]{%
	\texorpdfstring{\addtocounter{ecpoints\arabic{pppc}}{#1}\ec \disppoints{#1}}{}%
}
\newcommand{\mixedpoints}[2]{%
	\texorpdfstring{%
		\addtocounter{points\arabic{pppc}}{#1}%
		\addtocounter{ecpoints\arabic{pppc}}{#2}%
		(#1 (+#2) points)
	}{}%
}

% total points of current task         
\newcommand{\currentpoints}{% total points of current task   
	\texorpdfstring{\ifnumequal{\totvalue{ecpoints\arabic{pppc}}}{0}%
		{\total{points\arabic{pppc}} points}%
		{\total{points\arabic{pppc}}+\total{ecpoints\arabic{pppc}} points}}{}%
}


\def\squarebox#1{\hbox to #1{\hfill\vbox to #1{\vfill}}}
\def\qed{\hspace*{\fill}
	\vbox{\hrule\hbox{\vrule\squarebox{.667em}\vrule}\hrule}}
\newenvironment{solution}{\begin{trivlist}\item[]{\bf Solution:}}
	{\qed \end{trivlist}}
\newenvironment{solsketch}{\begin{trivlist}\item[]{\bf Solution
			Sketch:}} {\qed \end{trivlist}}
\newenvironment{code}{\begin{tabbing}
		12345\=12345\=12345\=12345\=12345\=12345\=12345\=12345\= \kill }
	{\end{tabbing}}


\newcommand{\hint}[1]{({\bf Hint}: {#1})}
% Put more macros here, as needed.
\newcommand{\room}{\medskip\ni}
\newcommand{\brak}[1]{\langle #1 \rangle}
\newcommand{\bit}[1]{\{0,1\}^{#1}}
\newcommand{\zo}{\{0,1\}}
\newcommand{\C}{{\cal C}}

\newcommand{\nin}{\not\in}
\newcommand{\set}[1]{\{#1\}}
\renewcommand{\ni}{\noindent}
\renewcommand{\gets}{\leftarrow}
\renewcommand{\to}{\rightarrow}
\newcommand{\assign}{:=}

\newcommand{\AND}{\wedge}
\newcommand{\OR}{\vee}
\renewcommand{\And}{\mbox{\bf and }}
\newcommand{\Or}{\mbox{\bf or }}

\newcommand{\For}{\mbox{\bf for }}
\newcommand{\To}{\mbox{\bf to }}
\newcommand{\DownTo}{\mbox{\bf downto }}
\newcommand{\Do}{\mbox{\bf do }}
\newcommand{\If}{\mbox{\bf if }}
\newcommand{\Then}{\mbox{\bf then }}
\newcommand{\Else}{\mbox{\bf else }}
\newcommand{\Elseif}{\mbox{\bf else if }}
\newcommand{\While}{\mbox{\bf while }}
\newcommand{\Repeat}{\mbox{\bf repeat }}
\newcommand{\Until}{\mbox{\bf until }}
\newcommand{\Return}{\mbox{\bf return }}
\newcommand{\Halt}{\mbox{\bf halt }}
\newcommand{\Swap}{\mbox{\bf swap }}
\newcommand{\Ex}[2]{\textrm{exchange } #1 \textrm{ with } #2}
\newcommand{\Nil}{\mbox{\bf nil}}
\newcommand{\In}{\mathsf{inOrder}}
\newcommand{\Post}{\mathsf{postOrder}}
\newcommand{\Pre}{\mathsf{preOrder}}
\newcommand{\Root}{\mathsf{root}}
\newcommand{\Parent}{\mathsf{parent}}
\newcommand{\Left}{\mathsf{left}}
\newcommand{\Right}{\mathsf{right}}
\newcommand{\Middle}{\mathsf{middle}}
\newcommand{\True}{\textbf{true}}
\newcommand{\False}{\textbf{false}}
\newcommand{\Print}{\mbox{\bf print }}
\newcommand{\ec}{({\bf Extra Credit})}
\newcommand{\note}{{\bf Note to Graders: }}
\newcommand{\Note}{\note}
\newcommand{\Rotate}{\textsc{Rotate}}
\newcommand{\LRotate}{\textsc{LeftRotate}}
\newcommand{\RRotate}{\textsc{RightRotate}}
 \newtheorem{definition}{Definition}
\newtheorem{lemma}{Lemma}
\newtheorem{corollary}{Corollary}
\DeclareMathOperator*{\argmax}{arg\,max}
\newcommand{\Gray}{\textsc{gray}}
\newcommand{\White}{\textsc{white}}
\newcommand{\Black}{\textsc{black}}

\begin{document}





\ifnum\me=0

% Collaborators (on a per task basis):
%
% Task 1: *********** INSERT COLLABORATORS HERE *********** 
% Task 2: *********** INSERT COLLABORATORS HERE *********** 
% etc.
%

\fi

\ifnum\me=1

\handout{PS \num}{\today}{Lecturer: Yevgeniy Dodis}{Due: \due}
{Solution {\em Sketches} to Problem Set \num}

\fi

\ifnum\me=2

\handout{PS \num}{\today}{Lecturer: Yevgeniy Dodis}{Due: \due}{Problem
  Set \num}

\fi




\newproblem{Greedy Topological Sort}{\currentpoints}


Let $G = (V,E)$ be a directed graph and let $n=|V|$ and $m=|E|$ denote
the number of vertices and edges, respectively.
Consider the following greedy algorithm for topological
sort of $G$: 
``Find a vertex $v$ with no outgoing
edges. If no such $v$ exists, output `cyclic'. Else put $v$ as the
last vertex in the topological sort, remove $v$ from $G$ (by also
removing all incoming edges to $v$), and recurse on the remaining
graph $G'$ on $(n-1)$ vertices. For the empty graph, end the algorithm.''.



In a first part, we will prove the above algorithm to be correct.

\begin{subtasks}
	
	\item \points{3}
	Let $G$ to be acyclic with $n > 0$, i.e., non-empty. Show that $G$ has at least
	one vertex $v$ having no outgoing edges. 
	\hint{Do a proof by contradiction.}
	
	\ifnum\me<2
\begin{solution}   INSERT YOUR SOLUTION HERE   \end{solution}
	\fi
	
	
	\item \points{2}
	Now prove that the algorithm produces output ``cyclic'' if an only if there 
	is no topological sorting. \hint{Use part~(a) as part of an induction proof over the number of vertices $n$.}
	
	\ifnum\me<2
\begin{solution}   INSERT YOUR SOLUTION HERE   \end{solution}
	\fi
	
	
	\item \points{2}
	Now assume $G$ to be acyclic. Conclude the correctness proof by showing that the algorithm outputs a valid topological sorting.
	
	\ifnum\me<2
\begin{solution}   INSERT YOUR SOLUTION HERE   \end{solution}
	\fi
\end{subtasks}


In the following, we want to efficiently implement to above algorithm.
To this end, assume $G = (V,E)$ to be given in adjacency-list format, where each
adjacency list stores the \emph{outgoing} edges.

\begin{subtasks}
	\item \points{3}
	Show how to implement to above algorithm in time $O(n (m+n))$.
	\hint{Assume that you can set flags $v.color$ on each node $v \in V$ indicating whether the 
		node has already been output, i.e., removed from the graph. Do not try to explicitly construct $G'$ for the recursive invocation.} 
	
	\ifnum\me<2
\begin{solution}   INSERT YOUR SOLUTION HERE   \end{solution}
	\fi
	
	
	\item \points{4}
	Now assume that the graph $G$ is given in an adjacency-list format that stores the \emph{incoming} edges instead of the outgoing ones. (Converting between those two formats can be done in time $O(n+m)$.) 
	Show how to implement the algorithm in time $O(n^2)$ in this case. 
	\hint{You may want to preprocess the graph and store some information at each node that then gets updated through the algorithm.} 
	
	\ifnum\me<2
\begin{solution}   INSERT YOUR SOLUTION HERE   \end{solution}
	\fi
\end{subtasks}



\newproblem{Cool Orderings}{\currentpoints}

Recall that in the algorithm for computing the strongly connected
components (SCCs) of a graph $G = (V,E)$, the second DFS on the transpose
$G^T$ (i.e., on the graph with reversed edges) processes the nodes in decreasing order of their finishing time (w.r.t.\ the first DFS).

Define an ordering $v_1,\ldots,v_n$ of the vertices of $G$ to be
\emph{cool} if
\begin{itemize}
	\item the vertices can be split up into $k$ groups corresponding to
	the $k$ SCCs of $G$ by appropriately inserting $k-1$ separators,
	i.e., if there exist $i_1,\ldots,i_{k-1}$ such that the SCCs are
	$\set{v_1,\ldots,v_{i_1}},\set{v_{i_1+1},\ldots,v_{i_2}},\ldots,
	\set{v_{i_{k-1}+1},\ldots,v_n}$, and
	\item the $k$ groups are in topological order (w.r.t.\ the SCC
	graph).
\end{itemize}
For example, assume $G$ has vertices $V = \set{a,b,c,d,e}$ and edges
$E= \set{(a,c),(c,a),(a,d)} \cup \set{(b,e),(e,b),(b,d)}$.  Then, $G$
has SCCs $C_1 = \set{a,c}$, $C_2 = \set{b,e}$, and $C_3 = \set{d}$.
Moreover, the SCC graph of $G$ has edges $\set{(C_1,C_3),(C_2,C_3)}$.
Hence, $a,c,e,b,d$ and $b,e,a,c,d$ are examples of cool orderings,
whereas $b,a,c,d,e$ and $a,b,c,d,e$ are uncool.

\begin{subtasks}
	\item \points{3}
	Prove or disprove: sorting the vertices in decreasing order of their finishing time (of an arbitrary DFS) is always cool.
	
	\ifnum\me<2
\begin{solution}   INSERT YOUR SOLUTION HERE   \end{solution}
	\fi
\end{subtasks}
In the following, we want to prove that if the DFS is run on $G^T$ 	w.r.t.\ some cool ordering, the resulting DFS trees are the SCCs of $G$. To this end, we adapt the proof from the book accordingly.
Recall that $f(v)$ denotes the finishing time of node $v$ and for a component $U \subseteq V$, $f(U) = \max_{v \in U} f(v)$ denotes the last finishing time of any node in the component.

\begin{subtasks}
	\item \points{4} 
	Adapt the following lemma from the book to be about cool orderings instead of finishing times and prove the resulting lemma.
	\begin{lemma}[Lemma 20.14]
		Let $C$ and $C'$ be distinct strongly connected components in directed graph $G = (V,E)$. Suppose that there is an edge $(u,v) \in E$, where $u \in C'$ and $v \in C$. Then $f(C') > f(C)$.
	\end{lemma}
	\hint{Replace the function $f$ with an appropriate function $g$ and adjust the lemma where as needed.}
	
	\ifnum\me<2
\begin{solution}   INSERT YOUR SOLUTION HERE   \end{solution}
	\fi
\end{subtasks}

\noindent
Assuming you have solved subtask (b) correctly, one can adapt Corollary 20.15 and Theorem 20.16 verbatim (replacing $f$ with $g$) and obtain a proof that using a cool ordering indeed works for finding the SCCs. 






\newproblem{Flights of Fancy}{\currentpoints}


You are hired as the technical
consultant for an airlines company
called ``Untied'' to help improve
their connectivity and also their
profit margins. You are given
a weighted, undirected graph $G=(V,E)$ where
each node represents the cities
where Untied serves and an
edge $(u,v)$ denotes that there
is a flight service from $u$
to $v$ and from $v$ to $u$. Let $|V|=n,|E|=m$. 
The graph
need not be connected. In
parts (a)-(e), the weight corresponds
to the cost of operating that route. 

In a preliminary survey, you identify
a potential problem that is causing
reduced profit: Some customers have managed to find
hacker-fares which exploit indirect routings. That is,
they manage to bypass find cheaper fares from $u$ to $v$
(where a direct flight exists) via a sequence of one or more
intermediate stops. Therefore, you are asked to figure out 
whether your route network is a \emph{spanning tree}. 

\begin{subtasks}
	\item \points{5}
	Present an $O(n)$ algorithm where $|V|=n$ to determine if
	the given graph $G$ is a spanning tree
	or not. More formally,
	
	\begin{center}
		Input: A weighted, undirected graph $G=(V,E)$
		
		Output: Print $\True$ iff $G$ is a spanning tree,
		else print a possible cycle, i.e., the vertices in this cycle
		
		Constraint: $O(n)$-time algorithm
		
	\end{center}
	
	Clearly, describe your
	algorithm. Do not forget to
	argue the correctness and prove the running time. 
	
	You can invoke any algorithm discussed in
	the lecture, as a black-box. You can also
	modify these algorithms. Clearly state these
	modifications and justify the impact of
	these changes on the running time and correctness
	of the algorithm. 
	\hint{It is easy to define an $O(m+n)$
		algorithm that works even for directed graphs. However,
		here we want you to get an algorithm that
		runs in time $O(n)$ which is a huge save
		if $m>>n$. This algorithm relies on
		the fact that $G$ is undirected.}
	\ifnum\me<2
\begin{solution}   INSERT YOUR SOLUTION HERE   \end{solution}
	\fi
\end{subtasks}

\noindent
Now, you have determined that $G$
is not a spanning tree but $G$ is connected.
It makes sense to produce a spanning
tree whose weight is minimal. You can run
Kruskal's algorithm to produce such a spanning
tree. The pseudocode is provided below:

\begin{code}
	{\sc Kruskal}$(G,w)$\\
	1 \> $A=\emptyset$\\
	2 \> \For each vertex $v\in G.V$\\
	3 \> \> {\sc Make-Set}$(v)$\\
	4 \> sort the edges of $G.E$ into non-decreasing order by weight $w$\\
	5 \> \For each $(u,v)$ from the sorted list\\
	6 \> \> \If {\sc Find-Set}$(u)\neq\textsc{Find-Set}(v)$\\	
	7 \> \> \> $A=A\cup\{(u,v)\}$\\
	8 \> \> \> {\sc Uniion}$(u,v)$\\
	9 \> \Return $A$
\end{code}

\begin{subtasks}
	\item \points{6}
	In this question, you are asked to account for another constraint:
	routes cannot be removed willy-nilly, especially because
	of contracts.  
	
	Specifically, you are given
	this set of routes as $F=(V',E')$ which is \emph{a forest of edges}.
	Now, you are asked to produce the best
	possible spanning tree such that $F$ is always included
	in the resulting tree. More formally,
	
	\begin{center}
		Input: A weighted, connected, undirected graph $G=(V,E)$,
		a forest $F=(V',E')$ that is a sub-graph of $G$, i.e., $V' \subseteq V$ and $E' \subseteq E$.
		
		Output: Return the best possible spanning tree subject
		to the constraint below.
		
		Constraint: Each edge in $F$ is included in the
		spanning tree. 
		
	\end{center}
	
	
	Modify the pseudocode {\sc Kruskal}$(G,w)$ to
	{\sc Kruskal}$(G,F,w)$ that produces the new
	spanning tree. You will do this by inserting
	no more than five lines of code to the pseudocode
	provided earlier. Your algorithm should still run in
	$O(|E|\log |V|)$.
	
	Do not forget to argue the correctness and justify the running time. 
	\hint{Use a theorem from the book for correctness.}
	
	\ifnum\me<2
\begin{solution}   INSERT YOUR SOLUTION HERE   \end{solution}
	\fi
\end{subtasks}




\newproblem{Least Maximal-Weight Edge}{\currentpoints}

Recall from the lecture that the minimum-spanning-tree (MST) problem
is to find, in some undirected graph~$G$, a spanning sub-tree $T$ with
the smallest possible sum $\sum_{e\in T} w(e)$ of edge weights.
Consider the related problem of finding a spanning sub-tree $T'$ with
the least maximum edge weight $\max_{e\in T'} w(e)$.

\begin{subtasks}
	\item \points{4}
	Assume the edge weights are unique.  Argue why
	Kruskal's algorithm outputs a spanning tree $T$ that is also a
	(optimal) solution to the modified problem.  You may use the fact
	that Kruskal is a correct algorithm.
	
	\ifnum\me<2
\begin{solution}   INSERT YOUR SOLUTION HERE   \end{solution}
	\fi
	
	
	\item \mixedpoints{2}{2}
	Find a graph that contains a non-MST spanning tree $T$ with least maximum edge weight.  
	That is $T$ is a (optimal) solution to the modified problem but not an MST.  For
	\textbf{extra credit}, the edge weights must be unique.
	
	\ifnum\me<2
\begin{solution}   INSERT YOUR SOLUTION HERE   \end{solution}
	\fi
	
	\item \points{4} 
	Consider now graphs with edge weights that are not necessarily unique.  
	Show that any (optimal) solution $T$ to the	MST problem is also an (optimal) 
	solution to the modified problem.
	
	\noindent
	\hint{Do not use a specific algorithm's (such as Kruskal's or
		Prim's) correctness since they output some specific MST out of
		possibly many. Give a general argument instead.}
	
	\ifnum\me<2
\begin{solution}   INSERT YOUR SOLUTION HERE   \end{solution}
	\fi
\end{subtasks}





\end{document}
