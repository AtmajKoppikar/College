% me=0 student solutions (ps file), me=1 - my solutions (sol file),
% me=2 - assignment (hw file)
\def\me{0} \def\num{10} %homework number

\def\due{5 pm on Tuesday, November 28} %due date

\def\course{CSCI-GA.1170-001, 003 Fundamental Algorithms} %course name, changed only once

% **** INSERT YOUR NAME HERE ****
\def\name{Name}

% **** INSERT YOUR NETID HERE ****
\def\netid{NetID}

% **** INSERT NETIDs OF YOUR COLLABORATORS HERE ****
\def\collabs{NetID1, NetID2}


\iffalse

INSTRUCTIONS: replace # by the homework number.  (if this is not
ps#.tex, use the right file name)

Clip out the ********* INSERT HERE ********* bits below and insert
appropriate LaTeX code.  There is a section below for student macros.
It is not recommended to change any other parts of the code.


\fi
%

\documentclass[11pt]{article}

% ==== Packages ====
\usepackage{amsfonts,amsmath,amssymb}
\usepackage{mathtools}
\usepackage{latexsym}
\usepackage{etoolbox}
\usepackage{totcount}
\usepackage{fullpage}
\usepackage{graphicx}
\usepackage{tikz}
\usepackage{tikz-qtree}
\usepackage[bottom]{footmisc}
\usepackage{enumitem}
\usepackage{hyperref}

\usepackage{array}
\newcolumntype{C}[1]{>{\centering\let\newline\\\arraybackslash\hspace{0pt}}m{#1}}
\newcolumntype{R}[1]{>{\raggedright\let\newline\\\arraybackslash\hspace{0pt}}m{#1}}

\usetikzlibrary{arrows}

\setlength{\footskip}{1in} \setlength{\textheight}{8.5in}

\newcommand{\handout}[5]{
	\renewcommand{\thepage}{#1, Page \arabic{page}}
	\noindent
	\begin{center}
		\framebox{ \vbox{ \hbox to 5.78in { {\bf \course} \hfill #2 }
				\vspace{4mm} \hbox to 5.78in { {\Large \hfill #5 \hfill} }
				\vspace{2mm} \hbox to 5.78in { {\it #3 \hfill #4} }
				\ifnum\me=0
				\vspace{2mm} \hbox to 5.78in { {\it Collaborators: \collabs
						\hfill} }
				\fi
		} }
	\end{center}
	\vspace*{4mm}
}

\newcounter{pppp}
\newcounter{pppc}
\newcommand{\prob}{\arabic{pppp}} %problem number
\newcommand{\increase}{\stepcounter{pppp}\stepcounter{pppc}} %problem number

% Arguments: Title, Number of Points
\newcommand{\newproblem}[2]{
	\ifnum\me=0
	\ifnum\prob>0 \newpage \fi
	\increase
	\restartlist{subtasks}
	\setcounter{page}{1}
	\handout{\name{} (\netid), Homework \num, Problem \arabic{pppp}}
	{\today}{Name: \name{} (\netid)}{Due: \due}
	{Solutions to Problem \prob\ of Homework \num\ (#2)}
	\else
	\increase
	\restartlist{subtasks}
	\section*{Problem \num-\prob~(#1) \hfill {#2}}
	\fi
}

\newlist{subtasks}{enumerate}{1}
\setlist[subtasks]{label={(\alph*)},resume}

\newcounter{numpppp}
\loop
\stepcounter{numpppp}
% workaround bug in totcount (means \newtotcounter{points\arabic{pointsct}})
\begingroup%
\edef\tempcounter@@name{points\arabic{numpppp}}%
\expandafter\newtotcounter\expandafter{\tempcounter@@name}%
\edef\tempcounter@@name{ecpoints\arabic{numpppp}}%
\expandafter\newtotcounter\expandafter{\tempcounter@@name}%
\endgroup
\ifnum \value{numpppp}<500 % max number of tasks supported
\repeat

% formating of output
\newcommand{\disppoints}[1]{%
	\texorpdfstring{(#1~\ifnumequal{#1}{1}{point}{points})}{}
}

% adds and displays
\newcommand{\points}[1]{%
	\texorpdfstring{\addtocounter{points\arabic{pppc}}{#1}\disppoints{#1}}{}%
}
\newcommand{\ecpoints}[1]{%
	\texorpdfstring{\addtocounter{ecpoints\arabic{pppc}}{#1}\ec \disppoints{#1}}{}%
}

% total points of current task         
\newcommand{\currentpoints}{% total points of current task   
	\texorpdfstring{\ifnumequal{\totvalue{ecpoints\arabic{pppc}}}{0}%
		{\total{points\arabic{pppc}} points}%
		{\total{points\arabic{pppc}}+\total{ecpoints\arabic{pppc}} points}}{}%
}


\def\squarebox#1{\hbox to #1{\hfill\vbox to #1{\vfill}}}
\def\qed{\hspace*{\fill}
	\vbox{\hrule\hbox{\vrule\squarebox{.667em}\vrule}\hrule}}
\newenvironment{solution}{\begin{trivlist}\item[]{\bf Solution:}}
	{\qed \end{trivlist}}
\newenvironment{solsketch}{\begin{trivlist}\item[]{\bf Solution
			Sketch:}} {\qed \end{trivlist}}
\newenvironment{code}{\begin{tabbing}
		12345\=12345\=12345\=12345\=12345\=12345\=12345\=12345\= \kill }
	{\end{tabbing}}


\newcommand{\hint}[1]{({\bf Hint}: {#1})}
% Put more macros here, as needed.
\newcommand{\room}{\medskip\ni}
\newcommand{\brak}[1]{\langle #1 \rangle}
\newcommand{\bit}[1]{\{0,1\}^{#1}}
\newcommand{\zo}{\{0,1\}}
\newcommand{\C}{{\cal C}}

\newcommand{\nin}{\not\in}
\newcommand{\set}[1]{\{#1\}}
\renewcommand{\ni}{\noindent}
\renewcommand{\gets}{\leftarrow}
\renewcommand{\to}{\rightarrow}
\newcommand{\assign}{:=}

\newcommand{\AND}{\wedge}
\newcommand{\OR}{\vee}
\renewcommand{\And}{\mbox{\bf and }}
\newcommand{\Or}{\mbox{\bf or }}

\newcommand{\For}{\mbox{\bf for }}
\newcommand{\To}{\mbox{\bf to }}
\newcommand{\DownTo}{\mbox{\bf downto }}
\newcommand{\Do}{\mbox{\bf do }}
\newcommand{\If}{\mbox{\bf if }}
\newcommand{\Then}{\mbox{\bf then }}
\newcommand{\Else}{\mbox{\bf else }}
\newcommand{\Elseif}{\mbox{\bf else if }}
\newcommand{\While}{\mbox{\bf while }}
\newcommand{\Repeat}{\mbox{\bf repeat }}
\newcommand{\Until}{\mbox{\bf until }}
\newcommand{\Return}{\mbox{\bf return }}
\newcommand{\Halt}{\mbox{\bf halt }}
\newcommand{\Swap}{\mbox{\bf swap }}
\newcommand{\Ex}[2]{\textrm{exchange } #1 \textrm{ with } #2}
\newcommand{\Nil}{\mbox{\bf nil}}
\newcommand{\In}{\mathsf{inOrder}}
\newcommand{\Post}{\mathsf{postOrder}}
\newcommand{\Pre}{\mathsf{preOrder}}
\newcommand{\Root}{\mathsf{root}}
\newcommand{\Parent}{\mathsf{parent}}
\newcommand{\Left}{\mathsf{left}}
\newcommand{\Right}{\mathsf{right}}
\newcommand{\Middle}{\mathsf{middle}}
\newcommand{\True}{\textbf{true}}
\newcommand{\False}{\textbf{false}}
\newcommand{\Print}{\mbox{\bf print }}
\newcommand{\ec}{({\bf Extra Credit})}
\newcommand{\note}{{\bf Note to Graders: }}
\newcommand{\Note}{\note}
\newcommand{\Rotate}{\textsc{Rotate}}
\newcommand{\LRotate}{\textsc{LeftRotate}}
\newcommand{\RRotate}{\textsc{RightRotate}}
 \newtheorem{lemma}{Lemma}
\newcommand{\Gray}{\textsc{gray}}
\newcommand{\White}{\textsc{white}}
\newcommand{\Black}{\textsc{black}}

\begin{document}

\ifnum\me=0

% Collaborators (on a per task basis):
%
% Task 1: *********** INSERT COLLABORATORS HERE *********** 
% Task 2: *********** INSERT COLLABORATORS HERE *********** 
% etc.
%

\fi

\ifnum\me=1

\handout{PS \num}{\today}{Lecturer: Yevgeniy Dodis}{Due: \due}
{Solution {\em Sketches} to Problem Set \num}

\fi

\ifnum\me=2

\handout{PS \num}{\today}{Lecturer: Yevgeniy Dodis}{Due: \due}{Problem
  Set \num}

\fi


\newproblem{Failure Nodes via BFS}{10 points}

You have just assumed the role of network administrator for a financial company with offices across the globe. Your network is represented as an undirected, connected graph $G=(V,E)$ with $n$ servers connected by $m$ edges. 
You have determined that the graph is indeed connected. However, you realize that there are not enough redundancies to ensure recovery from failure. You have a limited budget so you decide that the best you can do is to avoid any single point of failure. A single point of failure is a server which when it goes down disconnects the network into different components.

\begin{subtasks}
	\item \points{3} 
	You want to test if a given node $w$ is
	a single point of failure. You will do this by modifying
	the standard BFS algorithm (reproduced below for your reference).
	Suitably modify the algorithm to write the pseudocode
	for {\sc IsSPOF-BFS}$(G,s,w)$ that returns $\True$ if $w$
	is a single point of failure and \False\ otherwise. 
	Here $s\neq w$ is an arbitrary node from which you begin
	exploring. 
	Do not forget to justify the correctness of your algorithm
	and state the runtime. 
	\begin{code}
		{\sc BFS}$(G,s)$\\
		1 \> \For each vertex $u\in$ $G.V-\{s\}$\\
		2 \> \> $u.color=\White$\\
		3 \> \> $u.d=\infty$\\
		4 \> \> $u.\pi=\Nil$\\
		5 \> $s.color=\Gray$\\
		6 \> $s.d=0$\\
		7 \> $s.\pi=\Nil$\\
		8 \> $Q=\emptyset$\\
		9 \> {\sc Enqueue}$(Q,s)$\\
		10 \> \While $Q\neq \emptyset$\\
		11 \> \> $u=\textsc{Dequeue}(Q)$\\
		12 \> \> \For each $v\in G.Adj[u]$\\
		13 \> \> \> \If $v.color==\White$\\
		14 \> \> \> \> $v.color=\Gray$\\
		15 \> \> \> \> $v.d=u.d+1$\\
		16 \> \> \> \> $v.\pi=u$\\
		17 \> \> \> \> {\sc Enqueue}$(Q,v)$\\
		18 \> \> $u.color=\Black$\\
	\end{code}
	
	\ifnum\me<2
\begin{solution}   INSERT YOUR SOLUTION HERE   \end{solution}
	\fi
	
	\item \points{2}
	Use Part (a) to identify all the single
	points of failure of graph $G$. What is the runtime of your algorithm?
	\ifnum\me<2
\begin{solution}   INSERT YOUR SOLUTION HERE   \end{solution}
	\fi
	
	\item \points{5}
	You are given a	node $u$ and are told that there exists
	a node $w$ such that $\delta(u,w)>n/2$. Here $\delta(u,v)$ denotes the 
	shortest distance (in the number of edges) between $u$ and $v$.
	This means that the two nodes are really far apart.
	You have an intuition that there
	might be a single point of failure that would
	disrupt communication between $u$ and $w$ given
	the large distance between them. 
	Specifically, you fear that there is a distinct
	server $v$ which breaks
	all communication paths between servers $u$ and $v$ when it
	crashes.
	
	Show that your intuition is right. In other words,
	show that there exists such a server $v$ which will
	disrupt all paths between servers $u$ and $w$ when it goes
	down. Give an algorithm running in time $O(m+n)$ 
	to find the node. 
	You	may assume a suitable graph representation and 
	also use any algorithm discussed in lecture in 
	a black box manner. Do not forget to justify
	the correctness of your algorithm and the runtime. 
	\hint{Run BFS from $u$.}
	\ifnum\me<2
\begin{solution}   INSERT YOUR SOLUTION HERE   \end{solution}
	\fi
\end{subtasks}


\newproblem{Tournaments}{\currentpoints}

A tournament is a directed graph $G=(V,E)$ built from a complete undirected graph
$H$ by assigning directions to each edge in $H=(V,E')$. In other words, 
for all $u,v\in V$ if $(u,v)\in E$ then $(v,u)\not\in E$. 

\begin{subtasks}
	\item \points{1}
	Show that adding any edge to $G$ would create a cycle. 
	\hint{In a directed graph, any path (of length larger than $0$) that starts and ends at the same node is a cycle.}
	
	\ifnum\me<2
\begin{solution}   INSERT YOUR SOLUTION HERE   \end{solution}
	\fi
	
	\item \points{6}
	Present a $O(|V|^2)$ recursive algorithm
	that prints a path that visits every vertex exactly
	once. Assume the graph to be given as an adjacency matrix.
	Precisely describe the algorithm either in words or using pseudocode. 
	Prove the correctness of your algorithm. Also, derive
	a recurrence relation for the worst-case running
	time of your algorithm and state its solution. 
	\hint{Consider a divide-and-conquer approach.}
	\ifnum\me<2
\begin{solution}   INSERT YOUR SOLUTION HERE   \end{solution}
	\fi
\end{subtasks}
Consider the following definition of $G=(V,E)$ where $V=\{1,2,\ldots,n\}$
and $E$ contains every edge of the form $(j,i)$ where $0<i<j\leq n$.
It is easy to see that $G$ is a valid tournament. 
\begin{subtasks}
	\item \points{1}
	Prove that $G$ is a directed acyclic graph (DAG). 
	\ifnum\me<2
\begin{solution}   INSERT YOUR SOLUTION HERE   \end{solution}
	\fi
	
	\item \points{3}
	Now observe that $G$ has one node of in-degree
	0 and this node is $n$. Characterize or draw the DFS Tree produced
	by the invocation {\sc DFS-Visit}$(n)$ assuming that the
	algorithm processes each neighbor of a node $v$ in:
	\begin{itemize}
		\item ascending order
		\item descending order
	\end{itemize}
	\ifnum\me<2
\begin{solution}   INSERT YOUR SOLUTION HERE   \end{solution}
	\fi
	
	\item \points{4}
	Now, assume that the version of DFS does not use
	any markings to avoid duplicate processing. In other words,
	each node is processed if it is reached. Now, let $T(n)$
	be the number of time that this version of DFS visits
	vertex 1 when {\sc DFS-Visit}$(n)$ is invoked.
	Derive a recurrence relation for $T(n)$. Show the
	necessary steps and solve the recurrence relation
	to get an exact value for $T(n)$. 
	\ifnum\me<2
\begin{solution}   INSERT YOUR SOLUTION HERE   \end{solution}
	\fi
\end{subtasks}






\newproblem{Roots of Cyclicity}{\currentpoints}

We can define the root $s$ of a directed graph $G=(V,E)$ 
as the node from which every other node is reachable.
In other words, $\forall\ v\in V$ there is 
a path from $s$ to $v$. Assume that $G$ does
not contain nodes which have both in-degree and out-degree 0. 

For each of the statement given below either prove
that the statement is true or construct a counterexample
to show that the statement is false. Your counterexample
should have the least number of nodes possible and at least
one directed edge. 

\begin{subtasks}
	\item \points{2} 
	Every DAG $G$ must contain a root. 
	\ifnum\me<2
\begin{solution}   INSERT YOUR SOLUTION HERE   \end{solution}
	\fi
	
	\item \points{2} 
	If $s$ is the root of a DAG, then it must have in-degree 0. 
	\ifnum\me<2
\begin{solution}   INSERT YOUR SOLUTION HERE   \end{solution}
	\fi
	
	\item \points{3}
	In a DAG, if there exists exactly one node $s$ of in-degree 0, then $s$ is the root. 
	\ifnum\me<2
\begin{solution}   INSERT YOUR SOLUTION HERE   \end{solution}
	\fi
	
	\item \points{2}
	Let us assume that $s$ is the root of $G$. 
	Then the following algorithm can be used to determine if $G$ is a 
	DAG:
	\begin{itemize}
		\item We begin with BFS starting at $s$. 
		\item If we ever hit a node twice, we have a cycle and the graph 
		is not a DAG.
		\item If we hit every node once, then the graph is a DAG. 
	\end{itemize}
	\ifnum\me<2
\begin{solution}   INSERT YOUR SOLUTION HERE   \end{solution}
	\fi
	
	\item \ecpoints{2}
	How would your answer to part (d) change if the graph $G$ was undirected?
	\ifnum\me<2
\begin{solution}   INSERT YOUR SOLUTION HERE   \end{solution}
	\fi
\end{subtasks}








\end{document}


